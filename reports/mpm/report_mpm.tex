%start by defining the document class
\documentclass[12pt]{article}

%below the package used for hyperlinking
\usepackage[a4paper, top=2.3cm, bottom=2.3cm, left=2.3cm, right=2.3cm]{geometry}
\usepackage{graphicx}	
\usepackage[brazil]{babel}
\usepackage{float}
\usepackage{hyperref}
\usepackage{amsfonts} 
\usepackage{amsmath}
\usepackage[brazil]{babel}
\graphicspath{ {../images/} }

\linespread{1.5}
\setlength{\parindent}{1.3cm}

\begin{document}

\begin{titlepage}
    \begin{center}
        \vspace*{1cm}
            
        \Huge
        \textbf{Simulação de Sinais Cerebrais de Espectroscopia por 
        Ressonância Magnética}
            
        \vspace{0.5cm}
        \LARGE
        Da Criação à Corrupção (Por Ruído)
            
        \vspace{1.5cm}
            
        \textbf{João Victor Dell Agli Floriano \\ Fernando Fernandes Paiva}

            
        \vfill
            
        \textbf{Curso:} Mestrado \\
        \textbf{Período a que se refere:} 02/2024 a 12/2024
            
        \vspace{0.8cm}
            
        \Large
        \textbf{Bolsa de Estudos:} CAPES \\
        \textbf{Período de Vigência:} 01/08/2024 a 28/02/2026 (19 meses)
            
    \end{center}
\end{titlepage}

%inserting the title defined above

\section{Resumo}

\section{Introdução}

\begin{enumerate}
    \item Implementação do MPM sem ruído
    \item Implementação do MPM com ruído
    \item Separação de variáveis (s0, phi, omega, T2)
    \item Testagem do L sem ruído
    \item Testagem do SVD sem ruído
    \item Testagem do L com ruído
    \item Testagem do SVD com ruído
    \item Testagem do comportamento das variáveis separadas com a introdução de ruído de valores de sigma variados
\end{enumerate}

O método de "lápis de matrizes", do inglês \textit{Matrix Pencil Method} (MPM) é uma técnica numérica
de estimativa de parâmetros de sinais, desenvolvido originalmente por Yingbo Hua e Tapan Sakar \cite{370583} como uma alternativa a métodos já existentes 
como o de Prony \cite{49090}. O método consiste em modelar os sinais como uma soma de exponenciais complexas amortecidas, como na \autoref{eq:1}. Partindo 
dessa ideia, é então aplicada uma série de etapas, que inclui a utilização de outros métodos, como Decomposição em Valores Singulares (SVD, do inglês \textit{Singular Value Decomposition}), 
para estimar os parâmteros dessa função modeladora.  

\begin{equation} \label{eq:1}
    y(n) = \sum_{k=1}^{M} R_k e^{i (\omega_k t + \phi_k) + \alpha_k }
\end{equation}

\section{Métodos}

Desenvolvido por Yingbo Hua e Tapan Sakar \cite{370583}, a implementação do MPM em seu trabalho é descrita originalmente de duas maneiras: a sem ruído, 
implementada de maneira mais simplificada; e a que leva em conta a presença de ruído, que utiliza algoritmos mais complexos em sua implementação, como a
Decomposição em Valores Singulares (SVD, do inglês \textit{Singular Value Decomposition}).

\subsection{Caso sem ruído}

Para o caso sem ruído, define-se duas matrizes $(N-L) \times L$, $Y_1$ e $Y_2$, descritas pela \autoref{eq:2} e \autoref{eq:3}.

\begin{equation} \label{eq:2}
    Y_2 = \begin{bmatrix} x(1) & x(2) & \dots & x(L) \\
                            x(2) & x(3) & \dots & x(L + 1) \\
                            \vdots & \vdots & & \vdots \\
                            x(N-L) & x(N - L + 1) & \dots & x(N-1) \\
    \end{bmatrix}
\end{equation}

\begin{equation} \label{eq:3}
    Y_1 = \begin{bmatrix} x(0) & x(1) & \dots & x(L-1) \\
                            x(1) & x(2) & \dots & x(L) \\
                            \vdots & \vdots & & \vdots \\
                            x(N - L - 1) & x(N - L) & \dots & x(N-2) \\
    \end{bmatrix}
\end{equation}

Sendo $L$ o parâmetro de \textit{pencil}, que, em etapas posteriores, se mostra eficiente em eliminar alguns dos efeitos do ruído nos dados.

É possível escrever $Y_1$ e $Y_2$ como:
\begin{equation} 
    Y_2 = Z_1 R Z_0 Z_2
\end{equation}

e

\begin{equation}
    Y_1 = Z_1 R Z_2
\end{equation}

Sendo:
\begin{equation}
    Z_1 = \begin{bmatrix} 1 & 1 & \dots & 1 \\
                            z_1 & z_2 & \dots & z_M \\
                            \vdots & \vdots & & \vdots \\
                            z_1^{N - L - 1} & z_2^{N - L - 1} & \dots & z_M^{N - L - 1} 
    \end{bmatrix}  
\end{equation}

\begin{equation}
    Z_2 = \begin{bmatrix} 1 & z_1 & \dots & z_1^{L-1} \\
                            1 & z_2 & \dots & z_2^{L-1} \\
                            \vdots & \vdots & & \vdots \\
                            1 & z_M & \dots & z_M^{L-1} \\
    \end{bmatrix}
\end{equation}

\begin{equation}
    Z_0 = diag(z_1, z_2, \dots, z_M)
\end{equation}

E, por fim:

\begin{equation}
    R = diag(R_1, R_2, \dots, R_M)
\end{equation}

\section{Resultados}

\section{Conclusão}


\bibliographystyle{plain}
\bibliography{refs}


%nding the document
\end{document}