%start by defining the document class
\documentclass{article} 

%below the package used for hyperlinking
\usepackage{graphicx}
\usepackage[brazil]{babel}
\usepackage{float}
\graphicspath{ {./images/} }


%Parameters for titles
\title{Projeto A: Sistema de Partículas}
\author{João Victor Dell Agli Floriano}
\date{08/10/2024}


%beggining of doc
\begin{document}

%inserting the title defined above
\maketitle

\section{Resumo}

Apesar de o formalismo de Fourier ter representado uma importante revolução na área de processamento de sinais, 
sendo uma das bases fundamentais de Ressonância Magnética Nuclear, ele apresenta limitações de identificação de 
sinais específicos em espectros com alto grau de superposição de picos. Nesse contexto, o Matrix Pencil Method (MPM) 
foi apresentado como uma alternativa de pós-processamento, visto que sua aplicação permite uma melhor separação das 
componentes de interesse do sinal.(1) Entretanto, o método ainda carece de limites bem definidos, particularmente no 
que se refere à aplicabilidade específica aos dados de Espectroscopia por Ressonância Magnética (MRS), nos quais o 
nível de ruído pode ser um desafio extra. Assim, o presente trabalho tem como objetivo central a implementação e 
utilização do MPM para processamento dos dados de MRS, bem como a avaliação de sua acurácia e eficiência. Para isso, 
foram implementadas rotinas para simulação de sinais sintéticos de MRS cerebral. Essa implementação foi realizada 
a partir de uma base de dados de metabólitos, compostos químicos encontrados no cérebro humano. (2) Esse sinal temporal 
complexo foi corrompido por alguns fatores que reconhecidamente influenciam na legibilidade do sinal.(3), sendo elas ruído, 
aqui gerado artificialmente, alargamento dos picos espectrais, através da alteração dos tempos de relaxação efetivos, 
e manutenção do sinal residual da água. O ruído foi gerado utilizando valores gerados aleatoriamente a partir de uma 
distribuição de probabilidade gaussiana. Cada uma das componentes de ruído é adicionada à uma componente do sinal complexo e, 
então, o sinal de magnitude é calculado. O espectro de MRS é, então, obtido a partir da Transformada de Fourier (FT) do 
sinal e, pode, então, ser analisado. Um estudo inicial está sendo realizado de maneira a caracterizar a relação entre a 
relação sinal ruído (SNR) do espectro simulado e o desvio padrão (SD) da distribuição geradora do ruído no sinal temporal. 
Os primeiros resultados sugerem uma relação aproximadamente linear para valores mais baixos de ruído e uma saturação 
para valores mais altos. Os limites destes comportamentos, bem como sua caracterização para diferentes condições do sinal 
simulado, ainda estão sendo completamente caracterizados. A partir de então, pretende-se, com este sinal sintético corrompido, 
estudar como se comporta o MPM sob diferentes níveis de corrupção, com o objetivo de delimitar melhor a eficiência 
deste algoritmo na obtenção dos espectros de interesse. Em seguida, pretende-se enfim testar o algoritmo em sinais in vivo 
de espectroscopia, usando os resultados obtidos na etapa anterior como base para uma discussão mais qualificada dos resultados.

\section{Introdução}

\begin{enumerate}
    \item Falar sobre a importância de se ter um ambiente controlado para o objetivo final do trabalho.
    \item 
\end{enumerate}

\section{Métodos}

Para atingir o objetivo final de avaliação do MPM em sinais cerebrais de MRS, algumas etapas foram 
necessárias para que as condições ideais de testagem fossem estabelecidas. Para que esse algoritmo 
seja devidamente avaliado, é necessário primeiro garantir um quantidade suficiente de sinais de 
MRS das mais variadas condições para que uma análise estatística adequada seja feita. 

Tendo esse objetivo como base, a primeira etapa do projeto foi a criação de um ambiente de simulação
que tivesse a capacidade de lidar com as demandas do projeto. A simulação, escrita em uma biblioteca 
própria customizada de python, parte da equação básica de modelagem de um sinal de ressonância magnética, 
descrita em \ref{eq:1}.

\begin{equation} \label{eq:1}
    M(t) = M_0 e^{i(\omega t + \phi)} e^{\frac{-t}{T_2}}
\end{equation}

Essa equação descreve o valor da magnetização da parte da amostra, medida em Tesla (T)

\begin{equation} \label{eq:2}
    SNR = \frac{P}{\sigma}
\end{equation}

\begin{equation} \label{eq:3}
    f(x) = \frac{1}{\sigma \sqrt{2\pi}}e^{-\frac{(x - \mu)^2}{2\sigma ^2}}
\end{equation}

Usando o Tetrametilsilano (TMS) como referência para os desvios químicos.


\begin{table}[H]
    \centering
    \begin{tabular}{|l|c|c|c|}
    \hline
    Metabolite & $\delta$ (ppm) & $T_2$ (s) & Amplitude (U.A.) \\
    \hline
    gaba & 1.9346 & 1.9900e-02 & 0.2917 \\
    naa & 2.0050 & 7.3500e-02 & 0.4289 \\
    naag & 2.1107 & 6.6000e-03 & 0.0290 \\
    glx2 & 2.1157 & 9.0900e-02 & 0.0184 \\
    gaba2 & 2.2797 & 8.3300e-02 & 0.0451 \\
    glu & 2.3547 & 1.1630e-01 & 0.0427 \\
    cr & 3.0360 & 9.2600e-02 & 0.2026 \\
    cho & 3.2200 & 1.1360e-01 & 0.0776 \\
    m-ins3 & 3.2570 & 1.0530e-01 & 0.0202 \\
    m-ins & 3.5721 & 1.4710e-01 & 0.0411 \\
    m-ins2 & 3.6461 & 2.2220e-01 & 0.0150 \\
    glx & 3.7862 & 4.5700e-02 & 0.1054 \\
    cr2 & 3.9512 & 4.0000e-02 & 0.2991 \\
    cho+m-ins & 4.1233 & 8.8000e-03 & 0.8244 \\
    \hline
    \end{tabular}
    \caption{Informações dos metabólitos}
\end{table}

\section{Resultados e Discussão}

\section{Conclusão}

A partir do estudo da relação entre desvios padrão e o SNR resultante, 
foi verificada uma relação linear para valores baixos de sigma, e uma 
saturação para valores mais altos. Também foram calculados 
parâmetros de ajuste dos dados, com os quais foi feita a correção dos 
valores de SNR de entrada, viabilizando a dedução de uma função que 
relacionasse diretamente sigma e SNR com uma margem de erro 
relativamente baixa para as intenções do projeto.


%nding the document
\end{document}

%to compile, use pdflatex [name of the file].tex or use the compiler of vscode