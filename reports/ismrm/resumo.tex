%start by defining the document class
\documentclass[12pt]{article}

%below the package used for hyperlinking
\usepackage[a4paper, top=2.3cm, bottom=2.3cm, left=2.3cm, right=2.3cm]{geometry}
\usepackage{graphicx}	
\usepackage[brazil]{babel}
\usepackage{float}
\usepackage{hyperref}
\usepackage{amsfonts} 
\usepackage{amsmath}
\usepackage[brazil]{babel}
\graphicspath{ {images/} }

\linespread{1.2}
\setlength{\parindent}{1.3cm}

\title{O MPM como Ferramenta para o Processamento de Dados Clínicos de Ressonância Magnética: Uma Avaliação da Viabilidade}
\author{FLORIANO, J. V. D. A. \\ Instituto de Física de São Carlos \and PAIVA, F. F. \\ Instituto de Física de São Carlos}

\begin{document}

\maketitle

\section{Resumo}
O seguinte trabalho tem como objetivo caracterizar o uso do algoritmo Matrix Pencil Method (MPM) aplicado à sinais clínicos de MRS. 
A partir de uma simulação implementada \textit{in loco}  baseada em parâmetros empíricos de metabólitos presentes no cérebro, foram gerados 
sinais sintéticos de MRS, os quais foram corrompidos de maneira artificial com ruído gaussiano de valores variados. Esses sinais 
corrompidos foram então pós-processados por meio do algoritmo do MPM, implementado com sucesso também de maneira local, baseando-se 
no trabalho de Yingbo Hua e Tapan Sakar \cite{370583}, permitindo a recuperação dos parâmetros originais do sinal, o conjunto de 
amplitudes $S_0$, as frequências $\omega$, as fases $\phi$, e os tempos de decaimento $T_2$. O comportamento desses parâmetros foi analisado 
para diferentes SNR, revelando um papel importante da amplitude $S_0$ na composição do sinal. Foi também estudado o comportamento de 
um pico único sob efeito de ruído, revelando um comportamento exponencial decrescente na quantidade de picos de valor relevante 
conforme o SNR crescia, com prevalência do pico único como maior valor dentre todos. Espera-se elucidar melhor o comportamento de 
sinais clínicos sob a ótica do MPM, com perspectiva de encontrar caminhos para a melhora na qualidade de sua análise.

\section{Introdução}
O método “Lápis de Matrizes", do inglês \textit{Matrix Pencil Method} (MPM) é uma técnica numérica de estimativa de parâmetros de 
sinais, desenvolvido originalmente por Yingbo Hua e Tapan Sakar \cite{370583} como uma alternativa a métodos já existentes como o 
de Prony \cite{49090}. O mesmo consiste em modelar os sinais como uma soma de exponenciais complexas amortecidas. Partindo dessa 
ideia, é então aplicada uma série de etapas, que inclui a utilização de outros métodos, como Decomposição em Valores Singulares 
(SVD, do inglês \textit{Singular Value Decomposition}), para estimar os parâmetros dessa função modeladora. Esse método, por sua 
vez, encontra na área de Ressonância Magnética Nuclear (do inglês, Nuclear Magnetic Resonance, NMR) uma potencial aplicabilidade, 
visto que o fenômeno segue o mesmo comportamento de sua função modeladora. Especificamente na subárea de Espectroscopia por 
Ressonância Magnética (do inglês, Magnetic Resonance Spectroscopy, MRS), o método pode oferecer uma alternativa ao problema de 
corrupção de sinais por ruído, que podem prejudicar significativamente a qualidade do sinal adquirido. O estudo desse fenômeno 
por meio do MPM pode revelar potenciais caminhos para sua resolução.

\section{Métodos} 
O MPM parte da modelagem de um sistema de matrizes a partir do conceito de “lápis de funções”. É construída uma matriz principal 
a partir do mesmo sinal, a qual cada linha será o sinal percorrido com um passo de diferença, e a distância total terá tamanho 
L, definindo também o tamanho do lado da matriz. Essa matriz é decomposta por meio do método de Decomposição em Valores Singulares 
(do inglês, Singular Value Decomposition, SVD), a qual serão filtrados os valores menos significativos da decomposição, correspondentes 
geralmente ao ruído, reconstruindo a mesma com menos elementos. Após a reconstrução, a matriz é separada em duas submatrizes, $Y_1$ e 
$Y_2$, a qual a primeira é a matriz principal sem a última coluna, e a segunda é a matriz principal sem a primeira coluna. Essas duas 
submatrizes, a partir de um lápis, tem então seus autovalores generalizados calculados, resultando nos polos $z_i$, que contém informações 
sobre a frequência $\omega_i$ e tempo de decaimento $T_{2, i}$. A partir dos polos, são então calculados os resíduos $R_i$ a partir da 
resolução de um problema de mínimos quadrados. Os resíduos, por sua vez, contém as informações restantes, sobre a amplitude $S_{0, i}$ 
e a fase $\phi_i$. Para estudo do algoritmo no contexto de ruído, sinais sintéticos de MRS foram corrompidos com ruído gaussiano de 
valores de desvio padrão correspondentes a relações sinal-ruído (do inglês, Signal-to-noise ratio, SNR) entre 1dB e 100dB. Esses 
sinais foram analisados por meio do algoritmo em diferentes etapas de testes.

\section{Resultados}
A implementação do MPM foi feita com sucesso, demonstrando capacidade de identificar picos individuais com precisão em sinais 
clínicos de MRS livres de ruído, a partir da separação de parâmetros. No caso dos sinais corrompidos, foi possível identificar 
uma série de frequências, das quais as pertencentes ao sinal original tinham seus valores ligeiramente distorcidos por conta do 
ruído. Analisando de maneira mais detalhada o comportamento dos parâmetros originais do sinal de maneira conjunta, averiguou-se 
que a amplitude $S_0$ desempenha um papel fundamental de filtro dos demais parâmetros, determinando o quanto cada frequência individual 
identificada comporá o sinal final. Foi também estudado o comportamento de um único pico, o N-acetyl-aspartato (NAA), em diferentes 
níveis de SNR. Esse estudo, por meio da separação de parâmetros, verificou que o número de picos individuais identificados pelo 
algoritmo de valor relevante, aqui escolhidos como maiores ou iguais a 5\% do pico de NAA (maior pico), demonstrou comportamento 
exponencialmente decrescente conforme o valor de SNR aumentou. Além disso, foram identificados vários picos sobrepostos na área 
correspondente ao pico do NAA.

\section{Conclusão}
Considerando sua capacidade de extração de informações de sinais pós-aquisição, o MPM se mostra uma interessante ferramenta 
para o estudo de sinais de MRS. Sua capacidade de separar os parâmetros individuais do sinal abre novos caminhos para o entendimento 
dos fenômenos na área, com a perspectiva de mitigar os efeitos associados à corrupção por ruído em sinais clínicos, possibilitando uma 
análise de maior qualidade. No tocante aos testes realizados, o parâmetro de amplitude, $S_0$, desempenha importante papel de filtro de 
parâmetros, demonstrando capacidade de seleção e modelagem do formato do sinal final. Com relação ao comportamento dos picos, foi 
possível verificar que, apesar do número de picos de valor significativo aumentar, o maior pico ainda se manteve, mesmo que com 
frequência distorcida. Como próxima etapa, pretende-se investigar o comportamento dos parâmetros do sinal com a sobreposição de 
dois picos.


\bibliographystyle{plain}
\bibliography{refs}


\end{document}