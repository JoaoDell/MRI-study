\documentclass[10pt]{article}
\usepackage[utf8]{inputenc}
\usepackage[T1]{fontenc}
\usepackage{lmodern}
\usepackage[a4paper, margin=2.5cm]{geometry}
\usepackage{ragged2e}
\usepackage{microtype} 
\begin{document}

% Cabeçalho idêntico ao PDF original
\centering
\textbf{20\textsuperscript{th} NMR USERS MEETING -- October, 06-10, 2025 -- Angra dos Reis, RJ, Brazil}

\noindent\rule{\textwidth}{0.4pt} % Linha apenas abaixo do título

\vspace{24pt}

\textbf{MATRIX PENCIL AS A TOOL FOR PROCESSING MAGNETIC RESONANCE CLINICAL DATA: A VIABILITY ASSESSMENT}

\vspace{12pt}

*João Victor Dell Agli Floriano\textsuperscript{1}, Fernando Fernandes Paiva\textsuperscript{1}

\vspace{6pt}

\textsuperscript{1}São Carlos Institute of Physics, University of São Paulo, Brazil; 

*joaovdafloriano@usp.br

\vspace{12pt}
\begin{flushleft}
\textbf{Keywords:} MRS, Signal Processing, MPM.
\end{flushleft}

\vspace{12pt}
\begin{justify}
\quad This work aims to evaluate the applicability of the Matrix Pencil Method (MPM) algorithm for processing clinical Magnetic Resonance Spectroscopy (MRS) data.

\

The MPM, based on the work of Yingbo Hua and Tapan Sarkar \cite{370583}, is a numerical technique for estimating signal parameters by modeling a matrix system derived from 
the concept of a "function pencil." A main matrix is constructed from the signal itself, where each row corresponds to the signal shifted by a fixed step difference, 
with the total shift length being L, which also defines the matrix size. This matrix is decomposed using the Singular Value Decomposition (SVD) method, and the least 
significant values of the decomposition—typically corresponding to noise—are filtered out, reconstructing the matrix with fewer elements. After reconstruction, the 
matrix is split into two submatrices, $Y_1$ and $Y_2$, where the former is the main matrix without its last column and the latter is the main matrix without its first column. 
These two submatrices, through a pencil, then have their generalized eigenvalues computed, resulting in the poles $z_i$, which contain information about the frequency $\omega_i$ 
and decay time $T_{2,i}$. From these poles, the residues $R_i$ are calculated by solving a least-squares problem. The residues, in turn, contain the remaining information about 
the amplitude $S_{0,i}$ and phase $\omega_i$.

\

Using an in-house simulation based on empirical parameters of brain metabolites, synthetic MRS signals were generated and artificially corrupted with Gaussian noise, 
varying the Signal-to-Noise Ratio (SNR) between 1 dB and 100 dB. These corrupted signals were then post-processed using the locally implemented MPM algorithm, 
successfully recovering the original signal parameters: the set of amplitudes $S_0$, frequencies $\omega$, phases $\phi$, and decay times $T_2$.

\

The behavior of these parameters was analyzed for different SNR levels, revealing an important role of the amplitude $S_0$ in signal composition. The performance of a 
single peak under noise was also studied, showing an exponentially decreasing trend in the number of relevant peaks as SNR increased, with the single peak prevailing 
as the highest value among all.

\

As a next step, it is intended to investigate the behavior of signal parameters with the overlap of two peaks, thereby increasing the method's potential applicability.
\end{justify}

\vspace{24pt}

\begin{flushleft}
% \textbf{REFERENCES}

\bibliographystyle{plain}
\bibliography{refs}

\vspace{6pt}

\vspace{12pt}

\end{flushleft}

\end{document}