\documentclass[10pt]{article}
\usepackage[utf8]{inputenc}
\usepackage[T1]{fontenc}
\usepackage{lmodern}
\usepackage[a4paper, margin=2.5cm]{geometry}
\usepackage{ragged2e}
\usepackage{microtype} 
\begin{document}

% Cabeçalho idêntico ao PDF original
\centering
\textbf{20\textsuperscript{th} NMR USERS MEETING -- October, 06-10, 2025 -- Angra dos Reis, RJ, Brazil}

\noindent\rule{\textwidth}{0.4pt} % Linha apenas abaixo do título

\vspace{24pt}

\textbf{AN NMR APPROACH TO UNDERSTAND CELLULOSE ETHER’S WATER RETENTION MECHANISM IN GLUE MORTAR}

\vspace{12pt}

*Luis Augusto Pereira\textsuperscript{1}, Agide Gimenez Marassi\textsuperscript{1}, Gustavo Mattos Fortes\textsuperscript{2}, Caroline Cassia Alves\textsuperscript{2}, Tito José Bonagamba\textsuperscript{1}

\vspace{6pt}

\textsuperscript{1}São Carlos Institute of Physics, University of São Paulo, Brazil; \textsuperscript{2}Saint-Gobain, Brazil

*lap\_scp@usp.br

\vspace{12pt}
\begin{flushleft}
\textbf{Keywords:} Cement mortar, cellulose ether, NMR relaxometry.
\end{flushleft}

\vspace{12pt}
\begin{justify}
\quad Cement-based materials, such as thin-set mortar adhesives, are globally utilized for their exceptional binding properties. While the fundamental mechanisms governing their performance are not yet fully understood, workability is known to be primarily controlled by two competing processes: hydration and water evaporation. [1] Hydration, the chemical reaction between cement and water, remains partially enigmatic despite recent advances, while evaporation represents the physical loss of unbound water from the system. Recent studies have demonstrated that cellulose ethers can effectively modulate these processes, particularly by reducing evaporation rates and modifying hydration kinetics. [2,3]

This study investigates how different cellulose ether types, hydroxypropyl methylcellulose (HPMC) and hydroxyethyl methylcellulose (HEMC), and concentrations (0.1, 0.2, and 0.7 wt\%) influence the performance of three industrial-grade thin-set mortar formulations provided by Saint-Gobain. Using low-field nuclear magnetic resonance (NMR) relaxometry, we monitor molecular-level water dynamics throughout the hydration process. Transverse relaxation time (T2) measurements quantify changes in water states. [4] Complementary one-dimensional high-field magnetic resonance imaging (1D MRI) tracks spatial water distribution in real time, providing insights into water migration and localization. [5]
\end{justify}

\vspace{24pt}

\begin{flushleft}
\textbf{REFERENCES}

\vspace{6pt}

\vspace{12pt}

\textbf{Acknowledgements:} The authors thank University of São Paulo and Saint-Gobain for support, research collaboration and financial assistance.
\end{flushleft}

\end{document}